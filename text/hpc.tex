Citações são essenciais para qualquer projeto de pesquisa. Por exemplo, no contexto de arquiteturas paralelas, uma boa referência é \cite{tanenbaum2016structured}. Para poder citar utilizando o comando \textit{cite}, preencha o arquivo \textit{bibliography.bib} com o devido registro no formato bibtex, que pode ser facilmente obtido em plataformas como o \textit{Google Scholar}.

Ressalta-se que a FAPESP não estabelece normas rígidas de formatação do projeto. Existem muitos pacotes latex interessantes que podem ser utilizados para embelezar o documento. Verifique, por exemplo, o pacote \textit{hyperref}, que pode ser utilizado para tornar links e referências internas (para figuras, tabelas, seções \textit{etc}) ``clicáveis'' e coloridos. Neste modelo, limitou-se a utilizar pacotes que foram utilizados no projeto submetido e aceito.